%-*-latex-*-
\documentclass[11pt]{article}
%\usepackage{extsizes}
\usepackage{fullpage,pictex}
\begin{document}
\title{Markov Chain Monte-Carlo using the Metropolis Algorithm: A Toy
Example} 
\author{Alan Rogers\thanks{Dept.\ of Anthropology, 1400E 270S,
University of Utah, Salt Lake City, UT 84112}}
\maketitle

\section{Program}
\begin{verbatim}
#include <stdio.h>
#include <stdlib.h>
#include <math.h>
#include <time.h>
/**
   Toy model of metropolis sampler.  Data is (are?) one toss of a
   coin, which comes up heads.  Likelihood of data is p, the
   probability of a head.  Prior density of p is uniform, so posterior
   density is

                     p
   Pr[p|heads] = -------- = 2p
                  1
                 int p dp
                  0 

   The goal is to reproduce this using Metropolis.  

   The proposal density is defined as follows:

   1. Draw an observation from a uniform distribution of width w,
   centered on zero, and add it to the current value of the Markov
   Chain.  Call the result y.

   2. If y > 1, then set z = 1 - (y-1).  (In other words, reflect at
   the boundary.)

   3. If y < 0, then set z = -y.  (Another reflection).

   Only one reflection will be needed to ensure that 0 <= z <= 1.
   This proposal distribution is symmetric: The probability of moving
   from x to y is the same as that of moving from y to x.
   Consequently, we can use the Metropolis algorithm without bothering
   with the correction introduced by Hastings.

   @author Alan Rogers <rogers@anthro.utah.edu>

   Revision History

   1998-3-27 Initial version.  Used a proposal distribution that was
   independent of current value of Markov Chain.  Proposal
   distribution was asymmetric so the Hastings correction was used.

   2001-7-13 Implemented a symmetric proposal distribution, dropped
   the Hastings correction.
**/
int main(void)
{
    double x=0.0; /* state variable with initial value */
    double y;     /* proposed value */
    double w = 0.4; /* width of proposal distribution */
    double delta; /* proposed change in x */
    int i, iterations=5000;
    int nacpt=0;    /* number of proposals accepted */
    double mr;
    double xsum=0.0;

    srand48((int) time((time_t *) 0));

    printf("%7s %10s %10s %10s", "it", "x", "mean", "nacpt");
    for(i=1; i<=iterations; i++) {
        /* delta is drawn from a uniform distribution of width w,
           symmetric about 0 */
        delta = w * (drand48() - 0.5);

        /* draw from proposal distribution */
        y = x + delta;
        if(y > 1.0)
            y = 2.0 - y;  /* reflect at upper boundary */
        if(y < 0.0)
            y = -y;       /* reflect at lower boundary */
        
        mr = y/x;             /* Metropolis ratio */

        if(mr >= 1.0 || drand48() <= mr) { /* accept */
            nacpt += 1;
            x = y;
        }
        xsum += x;
        if(i%10==0)
            printf("\n%7d %10.6f %10.6f %10d", i, x, xsum/i, nacpt);
    }
    printf("\n\nMean should converge to 2/3, density to 2*p\n");
    return 0;
}
\end{verbatim}

\section{Output}
\begin{verbatim}
     it          x       mean      nacpt
    100   0.375758   0.638317         92
    200   0.736181   0.665502        187
    300   0.679522   0.709999        282
    400   0.639146   0.707163        372
    500   0.128920   0.679922        466
  ......................................
   9500   0.925596   0.669464       8628
   9600   0.569342   0.670790       8724
   9700   0.936062   0.671021       8816
   9800   0.720008   0.669378       8907
   9900   0.803219   0.670840       9002
  10000   0.662465   0.670883       9087

Mean should converge to 2/3, density to 2*p
\end{verbatim}

\begin{figure}
\centering
\mbox{\beginpicture
\setcoordinatesystem units <0.01in,2in> point at 0 0
\setplotarea x from 0 to 500, y from 0 to 1
\axis bottom label {Iteration} ticks numbered from 0 to 500 by 100 /
\axis left label {$x$} ticks numbered from 0 to 1 by 1 /
\put {Panel A} [bl] at 3 1
\plot
      0          0
      1   0.044461
      2   0.150896
      3   0.150896
      4   0.216556
      5   0.319716
      6   0.319716
      7   0.314120
      8   0.152814
      9   0.152814
     10   0.152814
     11   0.291642
     12   0.126039
     13   0.288308
     14   0.123273
     15   0.051686
     16   0.054590
     17   0.204280
     18   0.271814
     19   0.271814
     20   0.406374
     21   0.578296
     22   0.737889
     23   0.735151
     24   0.813998
     25   0.947447
     26   0.774373
     27   0.656565
     28   0.747855
     29   0.716905
     30   0.622930
     31   0.727683
     32   0.876443
     33   0.860502
     34   0.723088
     35   0.660452
     36   0.835580
     37   0.910254
     38   0.845122
     39   0.941213
     40   0.941213
     41   0.869544
     42   0.899075
     43   0.899075
     44   0.941259
     45   0.879916
     46   0.944282
     47   0.996572
     48   0.959814
     49   0.871924
     50   0.748330
     51   0.840208
     52   0.922608
     53   0.932031
     54   0.900655
     55   0.933932
     56   0.930904
     57   0.971598
     58   0.787177
     59   0.676660
     60   0.866689
     61   0.802742
     62   0.750940
     63   0.819964
     64   0.966937
     65   0.986850
     66   0.939844
     67   0.939844
     68   0.873486
     69   0.965778
     70   0.849678
     71   0.849134
     72   0.957177
     73   0.967155
     74   0.967155
     75   0.805396
     76   0.824740
     77   0.798936
     78   0.663057
     79   0.755688
     80   0.872958
     81   0.927260
     82   0.874699
     83   0.905786
     84   0.743608
     85   0.774775
     86   0.622942
     87   0.563053
     88   0.553887
     89   0.497257
     90   0.652486
     91   0.787882
     92   0.709555
     93   0.804924
     94   0.886555
     95   0.963590
     96   0.896878
     97   0.933744
     98   0.801415
     99   0.760834
    100   0.774253
    101   0.749030
    102   0.940513
    103   0.975588
    104   0.856478
    105   0.856478
    106   0.888045
    107   0.999597
    108   0.929415
    109   0.877256
    110   0.888031
    111   0.891106
    112   0.893586
    113   0.910827
    114   0.910827
    115   0.959427
    116   0.828090
    117   0.952010
    118   0.915545
    119   0.954942
    120   0.986964
    121   0.918634
    122   0.778081
    123   0.801862
    124   0.858615
    125   0.896749
    126   0.721666
    127   0.759933
    128   0.827740
    129   0.645776
    130   0.810653
    131   0.968097
    132   0.820946
    133   0.689914
    134   0.639803
    135   0.805261
    136   0.949413
    137   0.858951
    138   0.963300
    139   0.935805
    140   0.796567
    141   0.852646
    142   0.944711
    143   0.965607
    144   0.965607
    145   0.876154
    146   0.845521
    147   0.998560
    148   0.998560
    149   0.930731
    150   0.884202
    151   0.977633
    152   0.993757
    153   0.797061
    154   0.822879
    155   0.625679
    156   0.760224
    157   0.836602
    158   0.830055
    159   0.983273
    160   0.926124
    161   0.781574
    162   0.649970
    163   0.803234
    164   0.696087
    165   0.766989
    166   0.628152
    167   0.740406
    168   0.912698
    169   0.889565
    170   0.904004
    171   0.890453
    172   0.951747
    173   0.912658
    174   0.981307
    175   0.972831
    176   0.878459
    177   0.940137
    178   0.940058
    179   0.805640
    180   0.999360
    181   0.915731
    182   0.829680
    183   0.806526
    184   0.838868
    185   0.990622
    186   0.990622
    187   0.990622
    188   0.987537
    189   0.836212
    190   0.757433
    191   0.910245
    192   0.931282
    193   0.913286
    194   0.923475
    195   0.773763
    196   0.920031
    197   0.920031
    198   0.989901
    199   0.989901
    200   0.893940
    201   0.795241
    202   0.971497
    203   0.915028
    204   0.799086
    205   0.889201
    206   0.923361
    207   0.825879
    208   0.936286
    209   0.974276
    210   0.954421
    211   0.778778
    212   0.742923
    213   0.867338
    214   0.699188
    215   0.524807
    216   0.616593
    217   0.467871
    218   0.361450
    219   0.414007
    220   0.472013
    221   0.472013
    222   0.389483
    223   0.570520
    224   0.658586
    225   0.768679
    226   0.870266
    227   0.888530
    228   0.977242
    229   0.870650
    230   0.934464
    231   0.907067
    232   0.814426
    233   0.938999
    234   0.995641
    235   0.815833
    236   0.733825
    237   0.585966
    238   0.634020
    239   0.633770
    240   0.659069
    241   0.725244
    242   0.680080
    243   0.854834
    244   0.681341
    245   0.567183
    246   0.655670
    247   0.700133
    248   0.767913
    249   0.747415
    250   0.620058
    251   0.650814
    252   0.770836
    253   0.823083
    254   0.884519
    255   0.960165
    256   0.789038
    257   0.912514
    258   0.944528
    259   0.969198
    260   0.875488
    261   0.689169
    262   0.630653
    263   0.678215
    264   0.868370
    265   0.827480
    266   0.844904
    267   0.769934
    268   0.701056
    269   0.850179
    270   0.991715
    271   0.811443
    272   0.955813
    273   0.854059
    274   0.854059
    275   0.658535
    276   0.631917
    277   0.597121
    278   0.735965
    279   0.841667
    280   0.685739
    281   0.769820
    282   0.853854
    283   0.952866
    284   0.967469
    285   0.907300
    286   0.892990
    287   0.803816
    288   0.887457
    289   0.811410
    290   0.805905
    291   0.846141
    292   0.713542
    293   0.725242
    294   0.898191
    295   0.958127
    296   0.843320
    297   0.742808
    298   0.902691
    299   0.960343
    300   0.985313
    301   0.868557
    302   0.687652
    303   0.831528
    304   0.886867
    305   0.962448
    306   0.966123
    307   0.966123
    308   0.824083
    309   0.824083
    310   0.791146
    311   0.839218
    312   0.689822
    313   0.736493
    314   0.902262
    315   0.970518
    316   0.911500
    317   0.901598
    318   0.883627
    319   0.954544
    320   0.998527
    321   0.819825
    322   0.633459
    323   0.469983
    324   0.359778
    325   0.431724
    326   0.493457
    327   0.504375
    328   0.425100
    329   0.557532
    330   0.453771
    331   0.562817
    332   0.481404
    333   0.305636
    334   0.355303
    335   0.385518
    336   0.260993
    337   0.084164
    338   0.166093
    339   0.166093
    340   0.166093
    341   0.166093
    342   0.304771
    343   0.179541
    344   0.179541
    345   0.179541
    346   0.200586
    347   0.200586
    348   0.050792
    349   0.033815
    350   0.104799
    351   0.104799
    352   0.104799
    353   0.006814
    354   0.041099
    355   0.027541
    356   0.026905
    357   0.038876
    358   0.069023
    359   0.041295
    360   0.162039
    361   0.162039
    362   0.162039
    363   0.308959
    364   0.180733
    365   0.180733
    366   0.172564
    367   0.147328
    368   0.052442
    369   0.052442
    370   0.097675
    371   0.097675
    372   0.140030
    373   0.140030
    374   0.235675
    375   0.109355
    376   0.300472
    377   0.414464
    378   0.414464
    379   0.584663
    380   0.584663
    381   0.448489
    382   0.519638
    383   0.693245
    384   0.565938
    385   0.589367
    386   0.715290
    387   0.910028
    388   0.760773
    389   0.648889
    390   0.791571
    391   0.692560
    392   0.742587
    393   0.631698
    394   0.551542
    395   0.442068
    396   0.411316
    397   0.530125
    398   0.528909
    399   0.530413
    400   0.339323
    401   0.378742
    402   0.456913
    403   0.391352
    404   0.355986
    405   0.368477
    406   0.280661
    407   0.331979
    408   0.290226
    409   0.399252
    410   0.399252
    411   0.208860
    412   0.208860
    413   0.208860
    414   0.334618
    415   0.379007
    416   0.265670
    417   0.112153
    418   0.114839
    419   0.114839
    420   0.260220
    421   0.394517
    422   0.584364
    423   0.514585
    424   0.631058
    425   0.790362
    426   0.921122
    427   0.947015
    428   0.971455
    429   0.904062
    430   0.971199
    431   0.904348
    432   0.780609
    433   0.629508
    434   0.571134
    435   0.571134
    436   0.684095
    437   0.647590
    438   0.524297
    439   0.434644
    440   0.539755
    441   0.542809
    442   0.616521
    443   0.498329
    444   0.590279
    445   0.784896
    446   0.833964
    447   0.978070
    448   0.874886
    449   0.941182
    450   0.985678
    451   0.901268
    452   0.870434
    453   0.928159
    454   0.749705
    455   0.631820
    456   0.607426
    457   0.714828
    458   0.665284
    459   0.676810
    460   0.807820
    461   0.942688
    462   0.872269
    463   0.707497
    464   0.855626
    465   0.656480
    466   0.495366
    467   0.495366
    468   0.629295
    469   0.734829
    470   0.905227
    471   0.964384
    472   0.836947
    473   0.754584
    474   0.865221
    475   0.811230
    476   0.884289
    477   0.998586
    478   0.898196
    479   0.816114
    480   0.648589
    481   0.666452
    482   0.837982
    483   0.734268
    484   0.908540
    485   0.904013
    486   0.898168
    487   0.945214
    488   0.970753
    489   0.835755
    490   0.818732
    491   0.732011
    492   0.827103
    493   0.733107
    494   0.620313
    495   0.737972
    496   0.776361
    497   0.787404
    498   0.863814
    499   0.725215
    500   0.602865
/
\setcoordinatesystem units <0.01in,2in> point at 0 1.5
\setplotarea x from 0 to 500, y from 0 to 1
\axis bottom label {Iteration} ticks numbered from 0 to 500 by 100 /
\axis left label {$\bar x$} ticks numbered from 0 to 1 by 1 
         withvalues {2/3} / at 0.67 / /
\put {Panel B} [bl] at 3 1
\setdots
\putrule from 0 0.67 to 500 0.67
\setsolid
\plot
      0        0
      1 0.044461
      2 0.097678
      3 0.115417
      4 0.140702
      5 0.176505
      6 0.200373
      7 0.216623
      8 0.208647
      9 0.202443
     10 0.197480
     11 0.206040
     12 0.199374
     13 0.206215
     14 0.200290
     15 0.190383
     16 0.181896
     17 0.183213
     18 0.188135
     19 0.192539
     20 0.203231
     21 0.221091
     22 0.244582
     23 0.265911
     24 0.288748
     25 0.315096
     26 0.332761
     27 0.344753
     28 0.359150
     29 0.371486
     30 0.379868
     31 0.391088
     32 0.406255
     33 0.420020
     34 0.428934
     35 0.435549
     36 0.446661
     37 0.459190
     38 0.469346
     39 0.481445
     40 0.492940
     41 0.502125
     42 0.511576
     43 0.520588
     44 0.530149
     45 0.537921
     46 0.546755
     47 0.556326
     48 0.564732
     49 0.571001
     50 0.574547
     51 0.579757
     52 0.586350
     53 0.592872
     54 0.598572
     55 0.604669
     56 0.610495
     57 0.616830
     58 0.619767
     59 0.620731
     60 0.624831
     61 0.627747
     62 0.629734
     63 0.632754
     64 0.637975
     65 0.643343
     66 0.647835
     67 0.652193
     68 0.655448
     69 0.659945
     70 0.662656
     71 0.665282
     72 0.669336
     73 0.673416
     74 0.677385
     75 0.679092
     76 0.681009
     77 0.682540
     78 0.682290
     79 0.683219
     80 0.685591
     81 0.688575
     82 0.690845
     83 0.693434
     84 0.694032
     85 0.694981
     86 0.694144
     87 0.692637
     88 0.691060
     89 0.688883
     90 0.688478
     91 0.689571
     92 0.689788
     93 0.691026
     94 0.693106
     95 0.695953
     96 0.698046
     97 0.700476
     98 0.701506
     99 0.702105
    100 0.702827
    101 0.703284
    102 0.705610
    103 0.708231
    104 0.709657
    105 0.711055
    106 0.712725
    107 0.715406
    108 0.717387
    109 0.718854
    110 0.720392
    111 0.721930
    112 0.723462
    113 0.725121
    114 0.726750
    115 0.728773
    116 0.729629
    117 0.731530
    118 0.733089
    119 0.734953
    120 0.737054
    121 0.738554
    122 0.738878
    123 0.739390
    124 0.740352
    125 0.741603
    126 0.741445
    127 0.741590
    128 0.742263
    129 0.741515
    130 0.742047
    131 0.743773
    132 0.744357
    133 0.743948
    134 0.743171
    135 0.743631
    136 0.745144
    137 0.745975
    138 0.747549
    139 0.748904
    140 0.749244
    141 0.749978
    142 0.751349
    143 0.752847
    144 0.754325
    145 0.755165
    146 0.755784
    147 0.757435
    148 0.759065
    149 0.760217
    150 0.761043
    151 0.762478
    152 0.763999
    153 0.764215
    154 0.764596
    155 0.763700
    156 0.763678
    157 0.764142
    158 0.764559
    159 0.765935
    160 0.766936
    161 0.767027
    162 0.766304
    163 0.766531
    164 0.766102
    165 0.766107
    166 0.765276
    167 0.765127
    168 0.766005
    169 0.766736
    170 0.767544
    171 0.768263
    172 0.769329
    173 0.770158
    174 0.771371
    175 0.772523
    176 0.773125
    177 0.774068
    178 0.775001
    179 0.775172
    180 0.776417
    181 0.777187
    182 0.777475
    183 0.777634
    184 0.777967
    185 0.779116
    186 0.780254
    187 0.781379
    188 0.782475
    189 0.782759
    190 0.782626
    191 0.783294
    192 0.784065
    193 0.784735
    194 0.785450
    195 0.785390
    196 0.786077
    197 0.786757
    198 0.787783
    199 0.788798
    200 0.789324
    201 0.789354
    202 0.790255
    203 0.790870
    204 0.790910
    205 0.791390
    206 0.792030
    207 0.792194
    208 0.792887
    209 0.793754
    210 0.794520
    211 0.794445
    212 0.794202
    213 0.794545
    214 0.794100
    215 0.792847
    216 0.792031
    217 0.790537
    218 0.788569
    219 0.786859
    220 0.785428
    221 0.784009
    222 0.782232
    223 0.781283
    224 0.780735
    225 0.780682
    226 0.781078
    227 0.781551
    228 0.782410
    229 0.782795
    230 0.783454
    231 0.783989
    232 0.784121
    233 0.784785
    234 0.785686
    235 0.785815
    236 0.785594
    237 0.784752
    238 0.784119
    239 0.783490
    240 0.782971
    241 0.782732
    242 0.782308
    243 0.782606
    244 0.782191
    245 0.781313
    246 0.780803
    247 0.780476
    248 0.780425
    249 0.780293
    250 0.779652
    251 0.779139
    252 0.779106
    253 0.779280
    254 0.779694
    255 0.780402
    256 0.780435
    257 0.780949
    258 0.781583
    259 0.782308
    260 0.782666
    261 0.782308
    262 0.781729
    263 0.781335
    264 0.781665
    265 0.781838
    266 0.782075
    267 0.782030
    268 0.781727
    269 0.781982
    270 0.782759
    271 0.782865
    272 0.783500
    273 0.783759
    274 0.784015
    275 0.783559
    276 0.783010
    277 0.782339
    278 0.782172
    279 0.782385
    280 0.782040
    281 0.781996
    282 0.782251
    283 0.782854
    284 0.783504
    285 0.783938
    286 0.784320
    287 0.784388
    288 0.784746
    289 0.784838
    290 0.784911
    291 0.785121
    292 0.784876
    293 0.784672
    294 0.785058
    295 0.785645
    296 0.785840
    297 0.785695
    298 0.786088
    299 0.786670
    300 0.787333
    301 0.787602
    302 0.787271
    303 0.787418
    304 0.787745
    305 0.788317
    306 0.788899
    307 0.789476
    308 0.789588
    309 0.789700
    310 0.789704
    311 0.789864
    312 0.789543
    313 0.789374
    314 0.789733
    315 0.790307
    316 0.790690
    317 0.791040
    318 0.791332
    319 0.791843
    320 0.792489
    321 0.792574
    322 0.792080
    323 0.791083
    324 0.789752
    325 0.788650
    326 0.787745
    327 0.786878
    328 0.785775
    329 0.785081
    330 0.784077
    331 0.783409
    332 0.782499
    333 0.781067
    334 0.779792
    335 0.778615
    336 0.777075
    337 0.775019
    338 0.773217
    339 0.771426
    340 0.769646
    341 0.767876
    342 0.766522
    343 0.764811
    344 0.763109
    345 0.761418
    346 0.759797
    347 0.758185
    348 0.756152
    349 0.754083
    350 0.752228
    351 0.750383
    352 0.748549
    353 0.746448
    354 0.744455
    355 0.742436
    356 0.740426
    357 0.738461
    358 0.736591
    359 0.734654
    360 0.733063
    361 0.731482
    362 0.729909
    363 0.728749
    364 0.727243
    365 0.725746
    366 0.724235
    367 0.722663
    368 0.720842
    369 0.719030
    370 0.717351
    371 0.715681
    372 0.714133
    373 0.712594
    374 0.711319
    375 0.709714
    376 0.708625
    377 0.707845
    378 0.707069
    379 0.706746
    380 0.706424
    381 0.705747
    382 0.705260
    383 0.705229
    384 0.704866
    385 0.704566
    386 0.704594
    387 0.705125
    388 0.705268
    389 0.705123
    390 0.705345
    391 0.705312
    392 0.705407
    393 0.705220
    394 0.704830
    395 0.704165
    396 0.703425
    397 0.702988
    398 0.702551
    399 0.702120
    400 0.701213
    401 0.700409
    402 0.699803
    403 0.699037
    404 0.698188
    405 0.697374
    406 0.696348
    407 0.695453
    408 0.694459
    409 0.693738
    410 0.693019
    411 0.691841
    412 0.690669
    413 0.689502
    414 0.688645
    415 0.687899
    416 0.686884
    417 0.685506
    418 0.684141
    419 0.682782
    420 0.681776
    421 0.681093
    422 0.680864
    423 0.680471
    424 0.680355
    425 0.680613
    426 0.681178
    427 0.681801
    428 0.682477
    429 0.682994
    430 0.683664
    431 0.684176
    432 0.684399
    433 0.684273
    434 0.684012
    435 0.683752
    436 0.683753
    437 0.683670
    438 0.683307
    439 0.682740
    440 0.682415
    441 0.682099
    442 0.681950
    443 0.681536
    444 0.681330
    445 0.681563
    446 0.681905
    447 0.682567
    448 0.682997
    449 0.683572
    450 0.684243
    451 0.684724
    452 0.685135
    453 0.685671
    454 0.685812
    455 0.685694
    456 0.685522
    457 0.685586
    458 0.685542
    459 0.685523
    460 0.685789
    461 0.686346
    462 0.686749
    463 0.686793
    464 0.687157
    465 0.687091
    466 0.686680
    467 0.686270
    468 0.686148
    469 0.686252
    470 0.686718
    471 0.687308
    472 0.687625
    473 0.687766
    474 0.688141
    475 0.688400
    476 0.688811
    477 0.689461
    478 0.689897
    479 0.690161
    480 0.690074
    481 0.690025
    482 0.690332
    483 0.690423
    484 0.690874
    485 0.691313
    486 0.691739
    487 0.692259
    488 0.692830
    489 0.693122
    490 0.693379
    491 0.693457
    492 0.693729
    493 0.693809
    494 0.693660
    495 0.693750
    496 0.693916
    497 0.694104
    498 0.694445
    499 0.694507
    500 0.694323
/
\endpicture}
\caption{A: State variable of Markov Chain.  B: Mean of chain (dots
  show equilibrium)} 
\end{figure}


\end{document}








